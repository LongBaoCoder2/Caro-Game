\section{Thư viện Pygame}
\subsection{Giới thiệu về thư viện Pygame}
SFML (Simple and Fast Multimedia Library) là một thư viện đa phương tiện được đóng góp từ nhiều người ở cộng đồng, được viết chủ yếu bằng ngôn ngữ C++.
Pygame là một thư viện của ngôn ngữ Python. Pygame tập hợp bộ mô- đun Python đa nền tảng được thiết kế để phát triển trò chơi.

\begin{figure}[H]
%\centering{\includegraphics[scale=0.1]{images/SFMLLogo}}
\caption{Logo của thư viện Pygame}
\end{figure}

Thư viện SFML có vài điểm tương đồng với thư viện SDL2 (Simple DirectMedia Layer 2), nhưng được viết chủ yếu theo phương pháp hướng đối tượng nên việc tiếp cận cho các phần mềm hướng đối tượng sẽ dễ dàng hơn nhiều so với SDL2.

Sử dụng thư viện SFML giúp ta viết được các chương trình có thể chạy trên nhiều nền tảng.

\subsection{Các modules của thư viện SFML}
Hiện tại, thư viện SFML cung cấp cho người dùng $5$ modules:
\begin{itemize}
\item \textbf{Audio:} cung cấp các lớp giúp xử lý về âm thanh như: phát một tập tin nhạc hoặc tập tin ghi âm...
\item \textbf{Graphics:} cung cấp các lớp giúp xử lý đồ họa như vẽ hình...
\item \textbf{Network:} cung cấp các lớp giúp xử lý các giao thức mạng nhưu HTTP, FTP...
\item \textbf{System:} cung cấp các lớp giúp xử lý các vấn đề hệ thống như thời gian, Unicode...
\item \textbf{Window:} cung cấp các lớp giúp xử lý cửa sổ sự kiện.
\begin{figure}[H]
%\centering{\includegraphics[scale=0.7]{images/sfmldetails}}
\caption{Chi tiết thư viện SFML. Nguồn: \url{gamedevspot.net}}
\end{figure}
\end{itemize}


\subsection{Các trang web về thư viện SFML}
Trang web chính thức của thư viện SFML, nơi người dùng có thể tìm kiếm tài liệu về các lớp của thư viện cũng như đặt các câu hỏi trong quá trình sử dụng: \url{https://www.sfml-dev.org/}.\\
GitHub của thư viện: \url{https://github.com/SFML/SFML}